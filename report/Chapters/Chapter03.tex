%************************************************
\chapter{Tensorflow basics}\label{ch:tensorflow_basics}

\ac{TF} is an open source software library for machine learning in various kinds of perceptual and language understanding tasks. It is currently used for both research and production by different teams in dozens of commercial Google products [\cite{DBLP:journals/corr/AbadiABBCCCDDDG16}], such as speech recognition, Gmail, Google Photos, and Google Search, many of which had previously used its predecessor DistBelief [\cite{40565}]. 

\href{https://www.tensorflow.org/}{TensorFlow} was originally developed by the Google Brain team for Google's research and production purposes and later released under the Apache 2.0 open source license on November 9, 2015.

It provides a Python API, as well as a less documented C++ API.

\section{Overview}

TensorFlowF is Google Brain's second generation machine learning system. While the reference implementation runs on single devices, \ac{TF} can run on multiple \acsp{CPU} and \acsp{GPU} (with optional \acs{CUDA} extensions for general-purpose computing on graphics processing units). It runs on 64-bit Linux or Mac OS X desktop or server systems, as well as on mobile computing platforms, including Android and Apple's iOS.

TensorFlow computations are expressed as stateful dataflow graphs. Many teams at Google have migrated from DistBelief to TensorFlow for research and production uses.

This library of algorithms originated from Google's need to instruct neural networks, to learn and reason similarly to how humans do, so that new applications can be derived which are able to assume roles and functions previously reserved only for capable humans; the name TensorFlow itself derives from the operations which such neural networks perform on multidimensional data arrays. These multidimensional arrays are referred to as "tensors" but this concept is not identical to the mathematical concept of tensors. Its purpose is to train neural networks to detect and decipher patterns and correlations.