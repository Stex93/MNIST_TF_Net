%************************************************
 \chapter{Introduction}\label{ch:introduction}
%************************************************

Machine learning is becoming more and more important in a huge number of domains. Softwares able to increase their performances with knowledge about the world are now used in smartphones as well as social networks, search engines and intelligent cars, too. At the same time, tasks are becoming increasingly hard and powerful models have to be developed.

Deep learning is a branch of machine learning based on a set of algorithms that attempt to model high level abstractions in data.
Research in this area attempts to make better representations and create models to learn these representations from large-scale unlabeled data. Some of the representations are inspired by advances in neuroscience and are loosely based on interpretation of information processing and communication patterns in a nervous system, such as neural coding which attempts to define a relationship between various stimuli and associated neuronal responses in the brain.

In this context neural networks, and in particular deep neural networks play a fundamental role. Various deep learning architectures such as convolutional deep neural networks, deep belief networks and recurrent neural networks have been applied to fields like computer vision, automatic speech recognition, natural language processing, audio recognition and bioinformatics where they have been shown to produce state-of-the-art results on various tasks. Actually, in many of these fields the amount of data to be processed is too large to use traditional Multi Layer FeedForward Neural Networks with success.

At the same time, a lot of different frameworks for deep learning have been released in the last few years. For instance, in 2014 Berkeley Vision and Learning Center released Caffe [\cite{jia2014caffe}]. Theano [\cite{2016arXiv160502688full}] is a numerical computation library for Python developed by a machine learning group at the Université de Montréal. Microsoft, too, developed CNTK,  its own deep learning framework. Among all these tools Google's TensorFlow deserves a special mention. 
